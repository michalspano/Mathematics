\newcommand{\rel}{\mathcal{R}}

A relation $\rel$ on $A$, that is, \(\rel \subset A \times A\), is a set of
ordered pairs of elements of $A$. Written in conventional set notation, the
relation $\rel$ is

\[
  \{(a, b) \in \rel: a \text{ related to } b\}
\]

The phrase ``$a$ is related to $b$'' can be interpreted in many ways, depending
on the problem at hand. Consider the following trivial example.

Example: Let $\rel$ be a relation on \(N = \{\text{``Ave'', ``Bob'',
``Caesar''}\}\). $\rel$ is defined by a pair of elements (in this case, names)
if the number of letters in the first name is greater than the number of
letters in the second name. Firstly, we examine \(N \times N\)

\begin{align*}
  N \times N = \{&(\text{Ave}, \text{Ave}), (\text{Ave}, \text{Bob}), (\text{Ave}, \text{Caesar}),\\
                 &(\text{Bob}, \text{Ave}), (\text{Bob}, \text{Bob}),
                 (\text{Bob}, \text{Caesar}),\\
                 &(\text{Caesar}, \text{Ave}), (\text{Caesar}, \text{Bob}),
                  (\text{Caesar}, \text{Caesar})\}
\end{align*}

However, given all possible pairs of names, we can now select the pairs that
satisfy the relation $\rel$ and receive

\[
  \rel = \{(\text{Caesar}, \text{Ave}), (\text{Caesar}, \text{Bob})\}
\]

\subsection{Equivalence Relations}

TODO: define what an equivalence relation means.

\subsection{Equivalence Class, Partition}

TODO: define what an equivalence class is.

A \textbf{partition} of a set $A$ is a union of all its equivalence classes. Keep
in mind that the equivalence classes are non-overlapping (i.e. disjoint) and
non-empty. That is, a partition of a set $A$ must follow the following rules:

\begin{enumerate}
  \item \(\forall i: A_i \neq \emptyset\)
  \item \(\forall i, j: A_i \cap A_j = \emptyset, i \neq j\) \label{prop:disjoint}
  \item \(\bigcup_{i} A_i = A\)
\end{enumerate}

\exercise Show that (\ref{prop:disjoint}) holds.
