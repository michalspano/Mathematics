\section{Groups}

\newcommand{\la}{\langle}
\newcommand{\ra}{\rangle}
\newcommand{\op}{\oplus}
\newcommand{\oo}{\otimes}

\newcommand{\pset}[1]{\mathcal{P}(#1)}

The following section discusses \textbf{groups} as a central concept in algebra.
Furthermore, it illustrates the concept with examples and exercises.

\exercise Let $S$ be a non-empty set. Suppose a bijective function \(f: S
\longrightarrow S\). Then, suppose a set of all such bijective functions
labeled $B$. Decide if \(\la B, \circ \ra\) is a group.

\exercise Let $A$ be a non-empty set. Decide if \(\la \pset{A}, \Delta \ra\) is
a group. The operator $\Delta$ is defined by

\[
  A \Delta B = A \backslash B \cup B \backslash A
\] % TODO: solve this exercise.

\exercise Let \(\la G, \star \ra\), \(\la H, \ostar \ra\) be groups. We define
the Cartesian product of the two groups \(\la G, \star \ra \times \la H, \ostar
\ra\) with an operation $\ast$ as the following

\[
  \la G \times H, \ast \ra = (g_1, h_1) \ast (g_2, h_2) = (g_1 \star g_2, h_1
  \ostar h_2)
\]

% \exercise Suppose the set \(\Z_{p}\) where $p$ is a prime number. Show that
% every element from the set has a multiplicative inverse.
