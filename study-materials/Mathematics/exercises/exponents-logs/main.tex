\documentclass{homework}
\usepackage{amsfonts} 

%Paper credentials%
\author{Michal Špano}
\date{\today}
\title{Príklady: Exponenty a logaritmy \\ Exercises for exponents and logarithms}
\address{Bratislava}

\begin{document}
\maketitle

\begin{enumerate}
    
    \item SK: Nakresli grafy funkcií a grafy ich inverzných funkcií (EN: Draw the graphs of the given functions and the graphs of their corresponding inverse functions): \\
    \begin{enumerate}
        \item $f:y=\log_{0.5} (x-e) + 3$ \\
        \item $f:y=\big(\frac{1}{3}\big)^{x+2}-1$ \\
        \item $f:y=\left|2^{x+1}-5\right|$ \\ 
    \end{enumerate}
    
    \item SK: Urči definičný obor funkcií v $\mathbb{R}$ (EN: Determine the domain of the given functions in $\mathbb{R}$): \\
    \begin{enumerate}
        \item $f:y=\cfrac{1}{\log^{2} (x+1)}$ \\
        \item $f:y=\sqrt{\log (\log x)}$ \\
        \item $f:y=\log (\sqrt{x^{2}+2})$ \\
    \end{enumerate}
    
    \item SK: Riešte v $\mathbb{R}$: (EN: Solve for $x$ in $\mathbb{R}$) \\
    \begin{enumerate}
        \item $\bigg(\cfrac{3}{4}\bigg)^{x-1}.
        \sqrt{\bigg(\cfrac{4}{3}\bigg)^{x}}
        =\cfrac{9}{16}$ \\
        \item $\log_{\frac{1}{3}} (3-x) \geq -3$ \\
        \item $\log_3 x = 5 - \cfrac{4}{\log_3 x}$ \\
    \end{enumerate}
    
    \item SK: Vyjadri inverzné funckie daných funkcií (EN: Express the inverse functions): \\
    \begin{enumerate}
        \item $f:y=\big(\frac{1}{2}\big)^{-x+3}-2$ \\
        \item $f:y=\log_{\frac{1}{2}} (\frac{1}{2}+x)-5$ \\
        \item $f:y=2-\frac{1}{2}\log_{3} (x+1)$
    \end{enumerate}
    
\end{enumerate}
\end{document}
