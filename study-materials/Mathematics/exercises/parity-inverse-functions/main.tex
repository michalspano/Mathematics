\documentclass{homework}
\usepackage{amsfonts}
\usepackage{parskip} % Avoid using \\ (a.k.a. new line)

%Paper credentials%
\author{Michal Špano}
\date{\today}
\title{Párne/nepárne, inverzné funkcie,\\Even/odd, inverse functions}
\address{Bratislava}

\begin{document}
\maketitle

\begin{enumerate}
    % Exercise number 1    
    \item SK: Zistite, či sú funkcie \textbf{párne}, resp. \textbf{nepárne} (EN: Express the \textbf{parity} of given functions):
    \begin{enumerate}

        \item $f:y=\cfrac{\left|x\right|}{x^{5}-x}$
        \item $f:y=\cfrac{1}{\sqrt[^3]{x}}$ 
        \item $f:y=\cfrac{1}{\ln (x)}$ 
        \item $f:y=2^{x}+2^{-x}$ 
        \item $f:y=\sin{x} + \cos{x}$ 
        
        % Additional note
        \textit{$^{*}$}\textit{Funkcia môže byť ani nepárna, ani párna. A function can be neither odd nor even:}
        \href{https://moviecultists.com/can-a-function-be-neither-even-nor-odd}{\color{blue}{link}}.
    \end{enumerate}
    
    % Exercise number 2
    \item SK: Určte \textbf{inverzné funkcie} daných funkcií (EN: Express the \textbf{inverse functions} of given functions):
    \begin{enumerate}
        \item $f:y=\cfrac{x+2}{\frac{1}{2}x-2}$ 
        \item $f:y=\cfrac{1}{x^{2}-x}$
        \item $f:y=\ln (x+2)-10$
        \item $f:y=\sqrt[^5]{2x+11}$
        \item $f:y=(x+e)^{5}$
    \end{enumerate}
    
\end{enumerate}
\end{document}
